\chapter{Trabalhos Relacionados}\label{cap:trabalhos:relacionados}

\section{Plataforma Lattes}\label{sec:lattes}

A plataforma Lattes, concebida em agosto de 1999, representa a experiência do \gls{cnpq} na integração de bases de dados de currículos e de instituições da área de ciência e tecnologia em um único sistema de informações, padronizando em âmbito nacional o formato para coleta de informações curriculares e cuja importância atual se estende, não só às atividades do \gls{cnpq}, como também às ações de incentivo a outras instituições federais e estaduais \cite{Lattes}.


\section{iAraucária}\label{sec:iaraucaria}

A plataforma iAraucária, tem como propósito principal, uma busca avançada e eficiente de cientistas ou de grupos de trabalho especializados em diversas áreas de conhecimento. Com um cadastro completo de profissionais e pesquisadores, a plataforma permite que empresas e organizações, que já possuem uma ideia clara das competências ou especialidades que estão procurandoao, encontrem cientistas qualificados para atender às suas demandas específicas. Essa solução promove colaboração e inovação de forma ágil e assertiva.