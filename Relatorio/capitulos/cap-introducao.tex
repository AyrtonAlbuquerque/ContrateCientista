% ------------------------------------------------------------------------------------------- %
%                                          Introdução                                         %
% ------------------------------------------------------------------------------------------- %
\chapter{Introdução}\label{cap:introducao}

%Um texto curto apresentando o capítulo.
% Flavia

% ---------------------------------- Considerações Iniciais --------------------------------- %
\section{Considerações iniciais}\label{sec:consideracoesIniciais}

% fazer um levantamento com os profissionais responsáveis da utf para encontrar um cientista
% na instituição, para descrever a dificuldade disso

Relacionar os problemas da sociedade com as especialidades dos cientistas é um desafio para as instituições acadêmicas. 
Um dos meios de comunicação com os cientistas, é realizado a partir de emails, onde a sociedade expõe suas necessidades, 
e um profissional é responsável pela indicação dos especialistas. Atualmente, esse trabalho, é realizado 
na ferramenta CV Lattes, através da pesquisa por nome dos cientistas e/ou área de conhecimento. O profissional 
que realiza a pesquisa, precisaria saber a especialidade de cada um dos cientistas da academia, além de ter um conhecimento prévio 
em cada área para determinar qual especialidade melhor atenderá a demanda, para então, recomendar o cientista adequado.

Nossa proposta tem como objetivo otimizar o processo de busca de cientistas capacitados pela área de atuação. O objeto de pesquisa 
será um sistema, que realizará a busca de cientistas através da base do CV Lattes, correlacionar os dados por meio de uma Inteligência 
Artificial, e disponíbilizar atrás de um aplicativo para aparelhos celulares.

% ---------------------------------------- Objetivos ---------------------------------------- %
\section{Objetivos}\label{sec:objetivos}

Nesta seção são apresentados os objetivos gerais e específicos deste projeto.

% -------------------------------------- Objetivo Geral ------------------------------------- %
\subsection{Objetivo geral}\label{subsec:objetivoGeral}

Desenvolver uma plataforma capaz de proporcionar a um usuário a oportunidade de encontrar um cientista ou pesquisador capaz de atender à sua demanda tecnológica e facilitar o contato entre as partes.

% ---------------------------------- Objetivos Específicos ---------------------------------- %
\subsection{Objetivos específicos}\label{subsec:objetivosEspecificos}

\begin{itemize}
    \item Desenvolver uma \gls{api} capaz de atender e realizar requisições através de \gls{http}.
    \item Integrar as informações contidas na ferramenta CV Lattes à \gls{api} desenvolvida.
    \item Desenvolver uma aplicacão mobile capaz de habiltar usuários a visualizar e manipular dados apresentados.
    \item Integrar a aplicação mobile desenvolvida à \gls{api}.
    \item Construir uma base de dados para a plataforma.
    \item Integrar a \gls{api} à base de dados.
\end{itemize}

% -------------------------------------- Justificativa -------------------------------------- %
\section{Justificativa}\label{sec:justificativa}

Para alcançar os objetivos da plataforma proposta o desenvolvimento do projeto pode ser subdividido em três blocos básicos, sendo eles a criação de uma \gls{ui} que atenderá a necessidade de proporcionar aos usuários da plataforma uma maneira fácil e conveniente de visualizar os dados sobre os cientistas bem como filtrar as informações de acordo com suas preferências.

O próximo bloco consiste na criação de uma base de dados capaz de persistir informações pertinentes ao sistema, como dados de usuários por exemplo. Vale notar que tal base não tem como objetivo inicial armazenar dados relativos aos cientistas e pesquisadores, já que estes estão presentes na base de dados do CV Lattes, porém esta estratégia pode vir a ser adotada caso barreiras em relação a obtenção das informações contidas nessa base sejam incontornáveis. 

Por fim, é necessário o desenvolvimento de uma \gls{api} de \gls{rest} que servirá como ponte entre a interface de usuário e a base de dados, bem como o consumo das informações contidas no CV Lattes.

% ---------------------------------- Estrutura do Trabalho ---------------------------------- %
% \section{Estrutura do trabalho}\label{sec:estruturaTrabalho}

% A estrutura do trabalho contém uma relação dos capítulos e uma descrição sucinta do que cada um deles contém. Esta seção fornece uma visão geral do trabalho no sentido da sua estrutura em capítulos\footnote{Teste de nota de rodapé 2.}.

% \caixa{Atenção}{O OverLeaf está demorando muito para compilar o modelo com o Capítulo de Exemplos, que explica como usar o LaTeX. Assim, esse capítulo foi removido (está comentado para não compilar), mas há um arquivo chamado \texttt{exemploPDF.pdf}, na raiz do projeto, que contém esse capítulo de exemplos!}
