% ------------------------------------------------------------------------------------------- %
%                                          Introdução                                         %
% ------------------------------------------------------------------------------------------- %
\chapter{Introdução}\label{cap:introducao}

% ---------------------------------- Considerações Iniciais --------------------------------- %
\section{Considerações iniciais}\label{sec:consideracoesIniciais}

Com a crescente complexidade dos processos de inovação nas empresas, estas se viram obrigadas a buscar novas fontes de conhecimento, sendo as universidades um dos melhores ambientes possíveis para esta busca. Contudo, as relações entre universidade e empresa não estão livres de dificuldades, pois existe um contraste entre os objetivos que regem a pesquisa acadêmica e o processo de desenvolvimento industrial, sendo o primeiro com foco na publicidade dos resultados e o último por muitas vezes requer sigilo \cite{UnicampIE}. 

Relacionar os problemas da sociedade com as especialidades dos cientistas é um desafio para as instituições acadêmicas. Para isso é necessário compreender quais fatores são determinantes para a solução de um problema, que requer um grau elevado de inovação, bem como apontar o melhor cientista apto a resolvê-lo. Atualmente esse trabalho é realizado pela pessoa ou departamento responsável, no caso da \gls{utfpr} pelo \gls{direc}, através da ferramenta CV Lattes, onde a busca por nome dos cientistas e/ou área de conhecimento é feita. O profissional que realiza a pesquisa, precisa saber a especialidade de cada um dos membros da academia, além de ter um conhecimento prévio em cada área para determinar qual especialidade melhor atenderá a demanda, para então recomendar o cientista adequado.

Uma dificuldade encontrada constantemente pelo \gls{direc}, é não encontrar o cientista apto para resolver certa demanda. Um exemplo dessa dificuldade pode ser observada a partir da demanda a seguir. Uma empresa de telefonia iria mudar sua prestadora de serviço, e as candidatas possuíam tecnologias diferentes, então se viu necessário a consultoria para identificar qual das tecnologias atenderia melhor esta empresa. Esta demanda passou por sete cientistas do centro acadêmico, repassado de um para outro. A solicitante entrou em contato direto com o último pesquisador atribuído, e no fim, a consultoria não foi realizada.

Se tratando de novas tecnologias, a privacidade é um fator importante. Pensando nisso, a demanda será cadastrada de forma geral para que o cientista tenha um primeiro entendimento, e então a solicitante poderá seguir com um \gls{nda} para que o pesquisador se aprofunde, tanto nos materiais, quanto nos conhecimentos confidenciais que ambas as partes desejam compartilhar para determinado propósito, mas cujo uso generalizado desejam restringir.

Esta proposta tem como objetivo analisar quais os parâmetros exercem mais peso na escolha de um cientista para atendimento de uma demanda, bem como estabelecer um modelo que melhor correlacione os dados entre demandas e cientistas a fim de otimizar o processo de busca, assim como o facilitar a formação de parcerias universidade-empresa. Para isto propõem-se a criação de um \gls{mvp} que habilite o cadstro de demandas por parte de empresas, bem como o cadastro de cientistas que podem ter seus dados curriculares integrados a partir da plataforma Lattes.

% ---------------------------------------- Objetivos ---------------------------------------- %
\section{Objetivos}\label{sec:objetivos}

% -------------------------------------- Objetivo Geral ------------------------------------- %
\subsection{Objetivo geral}\label{subsec:objetivoGeral}

Analisar o impacto da aplicação de técnicas de dados conectados sobre o processo de formação de parcerias entre universidades e empresas através da construção de um \gls{mvp} que possibilite a rápida associação de uma demanda a um cientista apto a atendê-la.

% ---------------------------------- Objetivos Específicos ---------------------------------- %
\subsection{Objetivos específicos}\label{subsec:objetivosEspecificos}

\begin{itemize}
    \item Aplicar técnicas de dados conectados para a construção de modelos de filtragem e associação de demandas.
    \item Integrar dados sobre demandas e currículos contidos em bases de dados como o CV Lattes.
    \item Construir um \gls{mvp} que implemente os modelos construídos.
    \item Aplicar o \gls{mvp} como case na \gls{utfpr}.
    \item Comparar os resultados com as ferramentas atualmente utilizadas para atingir o mesmo objetivo.
\end{itemize}

% -------------------------------------- Justificativa -------------------------------------- %
\section{Justificativa}\label{sec:justificativa}

O objetivo específico é facilitar a relação universidade-empresa, que hoje, a plataforma Lattes e seu mecanismo de busca, já defasado, é uma das ferramentas capaz de nortear a busca de um cientista capaz de atender as necessidades inovativas das empresas. A ferramenta proposta, que busca beneficiar a academia e o setor privado, requer o estudo de como as demandas das empresas devem estar especificadas e como as atuais bases de dados de expertise de cientistas podem ser úteis para identificar possíveis grupos de pesquisa ou indivíduos que possam melhor atender estas demandas.

Para alcançar os objetivos propostos é necessário primeiramente compreender como os conceitos de dados conectados podem auxiliar na otimização do processo de vínculo entre uma empresa e a universidade, bem como também é necessário elencar quais os fatores tornam os métodos utilizados atualmente ineficientes e problemáticos.

Com as causas do problema em evidência, pode-se então construir um modelo que melhor represente como os dados relacionados as demandas se conectam com os dados relacionados aos cientistas, evidenciando quais parâmetros são mais relevantes para a sugestão de um candidato sobre outro. 

Por fim, para validar o modelo construído é necessário aplicá-lo através da construção de uma aplicação que facilite o cadastro de dados sobre demandas, sendo esta a principal diferença com outras ferramentas com objetivo similar, bem como o cadastro de cientistas cujas informações poderão ser integradas a partir de uma base de dados já consolidada como, por exemplo, o CV Lattes ou sintética.

% ---------------------------------- Estrutura do Trabalho ---------------------------------- %
% \section{Estrutura do trabalho}\label{sec:estruturaTrabalho}

% A estrutura do trabalho contém uma relação dos capítulos e uma descrição sucinta do que cada um deles contém. Esta seção fornece uma visão geral do trabalho no sentido da sua estrutura em capítulos\footnote{Teste de nota de rodapé 2.}.

% \caixa{Atenção}{O OverLeaf está demorando muito para compilar o modelo com o Capítulo de Exemplos, que explica como usar o LaTeX. Assim, esse capítulo foi removido (está comentado para não compilar), mas há um arquivo chamado \texttt{exemploPDF.pdf}, na raiz do projeto, que contém esse capítulo de exemplos!}
