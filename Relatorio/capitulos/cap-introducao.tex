% ------------------------------------------------------------------------------------------- %
%                                          Introdução                                         %
% ------------------------------------------------------------------------------------------- %
\chapter{Introdução}\label{cap:introducao}

% ---------------------------------- Considerações Iniciais --------------------------------- %
\section{Considerações iniciais}\label{sec:consideracoesIniciais}

Com a crescente complexidade dos processos de inovação nas empresas, estas se viram obrigadas a buscar novas fontes de conhecimento, sendo as universidades um dos melhores ambientes possíveis para esta busca. Contudo, as relações entre universidade e empresa não estão livres de dificuldades, pois existe um contraste entre os objetivos que regem a pesquisa acadêmica e o processo de desenvolvimento industrial, sendo o primeiro com foco na publicidade dos resultados e o último por muitas vezes requer sigilo \cite{UnicampIE}. 

Relacionar os problemas da sociedade com as especialidades dos cientistas é um desafio para as instituições acadêmicas. Para isso é necessário compreender quais fatores são determinantes para a solução de um problema, que requer um grau elevado de inovação, bem como apontar o melhor cientista apto a resolvê-lo. Atualmente esse trabalho é realizado pela pessoa ou departamento responsável, no caso da \gls{utfpr} pela \gls{direc}, através da ferramenta CV Lattes, onde a busca por nome dos cientistas e/ou área de conhecimento é feita. O profissional que realiza a pesquisa, precisa saber a especialidade de cada um dos membros da academia, além de ter um conhecimento prévio em cada área para determinar qual especialidade melhor atenderá a demanda, para então recomendar o cientista adequado.

Uma dificuldade encontrada constantemente pela \gls{direc}, é não encontrar o cientista ou laboratório apto para resolver certa demanda. Um exemplo dessa dificuldade pode ser observada a partir da demanda a seguir. Uma empresa de telefonia iria mudar sua prestadora de serviço, e as candidatas possuíam tecnologias diferentes, então se viu necessário a consultoria para identificar qual das tecnologias atenderia melhor esta empresa. Esta demanda passou por sete cientistas do centro acadêmico, repassado de um para outro. A solicitante entrou em contato direto com o último pesquisador atribuído, e no fim, a consultoria não foi realizada.

Esta proposta tem como objetivo implementar o uso de diferentes \gls{ia} e \gls{llm} na extração e associação de informações relevantes de demandas e laboratórios para melhor guiar o processo de escolha de um laboratório na formação de uma parceria empresa-universidade bem como realizar uma análise comparativa do desempenho das inteligências utilizadas. Para isto propõem-se a criação de um \gls{mvp} que habilite o cadastro de demandas por parte de empresas, bem como o cadastro de laboratórios.

% ---------------------------------------- Objetivos ---------------------------------------- %
\section{Objetivos}\label{sec:objetivos}

% -------------------------------------- Objetivo Geral ------------------------------------- %
\subsection{Objetivo geral}\label{subsec:objetivoGeral}

Analisar e comparar o desempenho de diferentes inteligências na extração e associação de informações de demandas e laboratórios com o intuito de facilitar a formação de parcerias entre universidades e empresas.

% ---------------------------------- Objetivos Específicos ---------------------------------- %
\subsection{Objetivos específicos}\label{subsec:objetivosEspecificos}

\begin{itemize}
    \item Implementar diversas inteligências para a extração de dadsos.
    \item Comparar o desempenho de cada inteligência na associação de dados.
    \item Construir um \gls{mvp} que possibilite a utilização destas inteligências.
\end{itemize}

% -------------------------------------- Justificativa -------------------------------------- %
\section{Justificativa}\label{sec:justificativa}

Para poder promover a inovação nem sempre é necessário reinventar a roda, pois muitas vezes a solução para um problema está em otimizar algum processo na cadeia de produção de uma solução. Neste caso, pretende-se otimizar o contato entre o mundo produtivo e a academia. Hoje este contato é um processo manual executado por uma pessoa responsável, no caso da \gls{utfpr} pelo \gls{direc}, que utilizando a plataforma Lattes e informações internas sobre os laboratórios e pesquisadores, tenta apontar a melhor opção para uma demanda.

O problema deste método é a ineficiência, inerente a qualquer processo manual, e a subjetividade, pois a escolha do melhor laboratório para uma demanda é baseada no conhecimento prévio do responsável pela busca. A proposta desta tese é apontar as possíveis inteligências e modelos de linguagem capazes de automatizar este processo.

Por não existir um modelo consolidado para este problema em específico esta tese propõe a construção de um sistema que possibilite a comparação de diferentes inteligências na extração e associação de informações, resultando no ranqueamento destes modelos e facilitando a escolha da inteligência a ser utilizada.

Como prova de conceito, propõe-se a construção de um \gls{mvp} que possibilite o uso destes modelos de linguagem, possibilitando a inserção de informações de demandas e laboratórios a serem analisados e apresentando o resultado das análises realizadas.

% ---------------------------------- Estrutura do Trabalho ---------------------------------- %
\section{Estrutura do trabalho}\label{sec:estruturaTrabalho}

Esta tese está dividida em seis capítulos e referências bibliográficas. O capítulo introdutório descreve o problema que se pretende resolver, bem como os objetivos e justificativas para a realização deste trabalho. 

O segundo capítulo, fundamentação teórica, apresenta os conceitos necessários para a compreensão do tema do trabalho.

O terceiro capítulo, trabalhos relacionados, apresenta projetos que se propoem a resolver o problema abordado e que possam servir de entendimento para a compreensão do tema.

O quarto capítulo, materiais e métodos, apresenta as ferramentas e tecnologias utilizadas no desenvolvimento e as atividades realizadas.

O quinto capítulo, resultados, descreve o sistema desenvolvido, o escopo no qual o trabalho se delimita, sua modelagem e implementação.

Por fim, o sexto capítulo, conclusão, decorre sobre os resultados obtidos, algumas das vantagens e desvantagens do sistema e sugestões para trabalhos aprimoramentos futuros.