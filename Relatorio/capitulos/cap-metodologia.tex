% ------------------------------------------------------------------------------------------- %
%                                         Metodologia                                         %
% ------------------------------------------------------------------------------------------- %
\chapter{Materiais e Métodos}\label{cap:materialemetodos}

\section{Materiais}\label{sec:materiais}

\subsection{CV Lattes}\label{subsec:lattes}

A plataforma Lattes, concebida em agosto de 1999, representa a experiência do \gls{cnpq} na integração de bases de dados de currículos e de instituições da área de ciência e tecnologia em um único sistema de informações, padronizando em ambito nacional o formato para coleta de informações curriculares e cuja importância atual se estende, não só às atividades do \gls{cnpq}, como também às ações de incentivo a outras instituições federais e estaduais \cite{Lattes}.

A plataforma também possui uma página web para busca textual dos currículos cadastrados em seu sistema bem como uma ferramenta para extração de dados de sua base denominado \textit{Lattes Extrator}, porém tal ferramenta só é disponibilizada às instituições que devem efetuar cadastro e solicitar acesso a mesma.

\subsection{PostgreSQL}\label{subsec:postgresql}

O PostgreSQL é um banco de dados relacional contributivo, ou seja, tem seu desenvolvimento em código aberto, o que garante mais liberdade no uso, além de permitir diferentes implementações de acordo com as necessidades, e ele utiliza a linguagem SQL como base \cite{Amazon}. Muitos dos contribuintes são voluntários, mas o projeto se sustenta com patrocínios de diversar empresas de todo o mundo. É um projeto da Universidade da Califórnia em Berkeley e tem mais de 35 anos de desenvolvimento ativo na plataforma central \cite{PostgreSQL}.

\subsection{Linguagem C{\#} {\&} .NET Framework}\label{subsec:csharp}

C{\#} é uma linguagem de programação, fortemente tipada e orientada a objetos desenvolvida pela Microsoft em julho de 2000 e sua sintaxe foi baseada no C++ porém contendo influências de outras linguagens como Java. A linguagem permite que desenvolvedores construam diversos tipos de aplicações de forma segura e robusta que são executadas sobre a plataforma .NET \cite{CSharp}.

.NET Framework é uma plataforma de desenvolvimento que possui um \gls{clr}, que gerência a execução de código. Possui também uma \gls{bcl}, oferencendo um amplo leque de classes para a construções de aplicações. A Microsoft, sua desenvolvedora, modelou a ferramenta para uso multi-plataforma, porém a ferramenta funciona melhor com o sistema operacional Windows \cite{CSharpDevelopment}.

\subsection{Flutter {\&} Dart}\label{subsec:flutterdart}

Flutter é uma estrutura que tem seu desenvolvimento em código aberto, disponível pelo Google. Com apenas um código, é possível construir aplicativos em multi-plataformas (Android/iOS), utilizando componentes nativos de cada plataforma \cite{Flutter}. A estrutura utiliza a linguagem Dart, que é assíncrona e muito semelhante a linguagem JavaScript \cite{Dart}.

\subsection{Docker}\label{subsec:docker}

Docker é um motor de código aberto que automatiza a implementação de aplicações dentro de containers. Esta ferramenta torna possível a criação de aplicações mais portáteis, de fácil construção e colaboração, reduzindo o tempo em que um código escrito seja testado, implementado e utilizado \cite{TheDockerBook}.

\section{Métodos}\label{sec:metodo}

% Os métodos definem, de certa maneira, um plano geral do trabalho, com as principais atividades realizadas durante seu processo de desenvolvimento. São apenas as atividades, o que será feito e o que se espera obter com as mesmas. O que é obtido com a realização dessas atividades está no \autoref{cap:resultados}.

% Os métodos são, basicamente, uma sequência de atividades realizadas para definir o sistema, modelar o problema e a solução, implementar a solução, testar e implantar essa solução. Essas atividades devem enfatizar a forma de uso dos materiais de acordo com o referencial teórico e como foi procedido no sentido de alcançar os objetivos do trabalho.
% Os métodos incluem os procedimentos utilizados para se alcançar o objetivo do trabalho. Assim, ele abrange o ciclo de vida do sistema, da identificação do problema à implantação da solução. A identificação pode incluir a definição dos requisitos por parte do usuário e/ou cliente definindo a proposta do sistema. A implantação pode incluir a forma de gerar os instaladores, os recursos e forma de instalação do sistema, a forma de manutenção e de descontinuidade do sistema.

% A definição das atividades, passos, ou procedimentos que compõem os métodos podem (ou mesmo deve) estarem baseados em autores. Esses autores, normalmente, estão relacionados à engenharia de software.

% O tempo verbal a ser utilizado na descrição dos métodos é o passado, considerando que trata-se de métodos que foram aplicados para a obtenção dos resultados a serem apresentados.

\subsection{Aplicação Mobile}\label{subsec:app}
% Flavia

\subsection{API REST}\label{subsec:apirest}
% Ayrton https://pepa.holla.cz/wp-content/uploads/2016/01/REST-API-Design-Rulebook.pdf

\subsection{Integrações}\label{subsec:integrar}
% Ayrton
