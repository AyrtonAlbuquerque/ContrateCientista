% ------------------------------------------------------------------------------------------------ %
%                                            Metodologia                                           %
% ------------------------------------------------------------------------------------------------ %
\chapter{Materiais e Métodos}\label{cap:materialemetodos}

A ênfase deste capítulo está em reportar o que e como será feito para alcançar o objetivo do trabalho. Este capítulo pode ser subdividido, inicialmente, em duas seções, sendo uma para os materiais e outra para os métodos.

\section{Materiais}\label{sec:materiais}

Materiais são as ferramentas, as tecnologias, os ambientes de desenvolvimento e outros que são utilizados para realizar as atividades desde a definição dos requisitos à implantação do sistema. Exemplos de materiais: linguagens de programação e de modelagem, banco de dados e seus gerenciadores, editores para análise e modelagem, ambiente e plataforma de desenvolvimento.

Cada um dos materiais pode ter uma subseção própria ou serem descritos em uma mesma seção. De qualquer forma, essa seção não precisa ser muito extensa, deve abranger apenas um conhecimento básico sobre cada um dos materiais e o que é mais relevante ou utilizado para o trabalho proposto. De maneira geral, não há necessidade de incluir informações históricas sobre os materiais. Centrar-se nos conceitos e particularidades mais relevantes para o trabalho. Exceto se necessário para o entendimento do objeto do trabalho ou considerado relevante para o tipo de pesquisa.

\section{Métodos}\label{sec:metodo}

Os métodos definem, de certa maneira, um plano geral do trabalho, com as principais atividades realizadas durante seu processo de desenvolvimento. São apenas as atividades, o que será feito e o que se espera obter com as mesmas. O que é obtido com a realização dessas atividades está no \autoref{cap:resultados}.

Os métodos são, basicamente, uma sequência de atividades realizadas para definir o sistema, modelar o problema e a solução, implementar a solução, testar e implantar essa solução. Essas atividades devem enfatizar a forma de uso dos materiais de acordo com o referencial teórico e como foi procedido no sentido de alcançar os objetivos do trabalho.
Os métodos incluem os procedimentos utilizados para se alcançar o objetivo do trabalho. Assim, ele abrange o ciclo de vida do sistema, da identificação do problema à implantação da solução. A identificação pode incluir a definição dos requisitos por parte do usuário e/ou cliente definindo a proposta do sistema. A implantação pode incluir a forma de gerar os instaladores, os recursos e forma de instalação do sistema, a forma de manutenção e de descontinuidade do sistema.

A definição das atividades, passos, ou procedimentos que compõem os métodos podem (ou mesmo deve) estarem baseados em autores. Esses autores, normalmente, estão relacionados à engenharia de software.

O tempo verbal a ser utilizado na descrição dos métodos é o passado, considerando que trata-se de métodos que foram aplicados para a obtenção dos resultados a serem apresentados.
