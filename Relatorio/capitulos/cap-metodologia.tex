% ------------------------------------------------------------------------------------------- %
%                                         Metodologia                                         %
% ------------------------------------------------------------------------------------------- %
\chapter{Materiais e Métodos}\label{cap:materialemetodos}

\section{Materiais}\label{sec:materiais}

\subsection{PostgreSQL}\label{subsec:postgresql}

O PostgreSQL é um banco de dados relacional contributivo, ou seja, tem seu desenvolvimento em código aberto, o que garante mais liberdade no uso, além de permitir diferentes implementações de acordo com as necessidades, e ele utiliza a linguagem SQL como base \cite{Amazon}. Muitos dos contribuintes são voluntários, mas o projeto se sustenta com patrocínios de diversas empresas de todo o mundo. É um projeto da Universidade da Califórnia em Berkeley e tem mais de 35 anos de desenvolvimento ativo na plataforma central \cite{PostgreSQL}.

\subsection{Linguagem C{\#} {\&} .NET Framework}\label{subsec:csharp}

C{\#} é uma linguagem de programação, fortemente tipada e orientada a objetos desenvolvida pela Microsoft em julho de 2000 e sua sintaxe foi baseada no C++, porém contendo influências de outras linguagens como Java. A linguagem permite que desenvolvedores construam diversos tipos de aplicações de forma segura e robusta que são executadas sobre a plataforma .NET \cite{CSharp}.

.NET Framework é uma plataforma de desenvolvimento que possui um \gls{clr}, que gerencia a execução de código. Possui também uma \gls{bcl}, oferecendo um amplo leque de classes para a construção de aplicações. A Microsoft, sua desenvolvedora, modelou a ferramenta para uso multi-plataforma, porém a ferramenta funciona melhor com o sistema operacional Windows \cite{CSharpDevelopment}.

\subsection{Python}\label{subsec:python}

% texto

\subsection{Flutter {\&} Dart}\label{subsec:flutterdart}

Flutter é uma estrutura com seu desenvolvimento em código aberto, disponível pelo Google. Com apenas um código, é possível construir aplicativos em multi-plataformas (Android/iOS), utilizando componentes nativos de cada plataforma \cite{Flutter}. A estrutura utiliza a linguagem Dart, assíncrona e muito semelhante à linguagem JavaScript \cite{Dart}.

\subsection{Docker}\label{subsec:docker}

Docker é um motor de código aberto que automatiza a implementação de aplicações dentro de contêineres. Esta ferramenta permite a criação de aplicações mais portáteis, de fácil construção e colaboração, reduzindo o tempo em que um código escrito seja testado, implementado e utilizado \cite{TheDockerBook}.

\subsection{BERTimbau}\label{subsec:bertimbau}

% texto

\section{Métodos}\label{sec:metodo}

\subsection{API REST}\label{subsec:rest}

% texto

\subsection{Marcação de Parte de Fala}\label{subsec:pos_tagging}

% https://web.stanford.edu/~jurafsky/slp3/ed3bookaug20_2024.pdf

\subsection{Extração de Palavras-Chave}\label{subsec:keyword_extraction}

% texto

\subsection{Similaridade do Cosseno}\label{subsec:cossine_similarity}

% texto