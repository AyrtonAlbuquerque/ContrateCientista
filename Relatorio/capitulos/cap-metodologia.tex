% ------------------------------------------------------------------------------------------- %
%                                         Metodologia                                         %
% ------------------------------------------------------------------------------------------- %
\chapter{Materiais e Métodos}\label{cap:materialemetodos}

\section{Materiais}\label{sec:materiais}

\subsection{PostgreSQL}\label{subsec:postgresql}

O PostgreSQL é um banco de dados relacional contributivo, ou seja, tem seu desenvolvimento em código aberto, o que garante mais liberdade no uso, além de permitir diferentes implementações de acordo com as necessidades, e ele utiliza a linguagem SQL como base \cite{Amazon}. Muitos dos contribuintes são voluntários, mas o projeto se sustenta com patrocínios de diversas empresas de todo o mundo. É um projeto da Universidade da Califórnia em Berkeley e tem mais de 35 anos de desenvolvimento ativo na plataforma central \cite{PostgreSQL}.

\subsection{Linguagem C{\#} {\&} .NET Framework}\label{subsec:csharp}

C{\#} é uma linguagem de programação, fortemente tipada e orientada a objetos desenvolvida pela Microsoft em julho de 2000 e sua sintaxe foi baseada no C++, porém contendo influências de outras linguagens como Java. A linguagem permite que desenvolvedores construam diversos tipos de aplicações de forma segura e robusta que são executadas sobre a plataforma .NET \cite{CSharp}.

.NET Framework é uma plataforma de desenvolvimento que possui um \gls{clr}, que gerencia a execução de código. Possui também uma \gls{bcl}, oferecendo um amplo leque de classes para a construção de aplicações. A Microsoft, sua desenvolvedora, modelou a ferramenta para uso multi-plataforma, porém a ferramenta funciona melhor com o sistema operacional Windows \cite{CSharpDevelopment}.

\subsection{Python}\label{subsec:python}

% texto

\subsection{Flutter {\&} Dart}\label{subsec:flutterdart}

Flutter é uma estrutura com seu desenvolvimento em código aberto, disponível pelo Google. Com apenas um código, é possível construir aplicativos em multi-plataformas (Android/iOS), utilizando componentes nativos de cada plataforma \cite{Flutter}. A estrutura utiliza a linguagem Dart, assíncrona e muito semelhante à linguagem JavaScript \cite{Dart}.

\subsection{Docker}\label{subsec:docker}

Docker é um motor de código aberto que automatiza a implementação de aplicações dentro de contêineres. Esta ferramenta permite a criação de aplicações mais portáteis, de fácil construção e colaboração, reduzindo o tempo em que um código escrito seja testado, implementado e utilizado \cite{TheDockerBook}.

\subsection{GPT}\label{subsec:gpt}

O \gls{gpt} é uma classe de modelos de linguagem desenvolvidos pela OpenAI, introduzido inicialmente em 2018, com base na arquitetura de transformadores. Ao longo de suas versões (GPT-1, GPT-2, GPT-3), o modelo evoluiu significativamente em capacidade e complexidade, culminando em sistemas com níveis de fluidez e coerência comparáveis ao humano.

O modelo utiliza pré-treinamento autoregressivo unidirecional, prevendo a próxima palavra em uma sequência textual considerando apenas os elementos anteriores. O treinamento do GPT utiliza aprendizado auto-supervisionado, eliminando a necessidade de anotações dos dados. \cite{Brown2020}.

Entre as características que diferenciam o GPT de outras abordagens, destaca-se sua capacidade de realizar tarefas em cenários de aprendizado de poucos ou nenhum exemplo (few-shot e zero-shot learning), particularmente com o GPT-3, que possui 175 bilhões de parâmetros, permitindo que o modelo entenda e execute instruções com base em descrições mínimas. \cite{Brown2020}.

\subsection{BERT}\label{subsec:bert}

O \gls{bert} é um modelo de linguagem baseado em transformadores desenvolvido pelo Google AI em 2018 e que revolucionou o campo do processamento de linguagem natural ao introduzir um treinamento bidirecional profundo, permitindo a captura de contextos linguísticos em ambas as direções (esquerda para direita e direita para esquerda) simultaneamente. \cite{Devlin2018}.

O terinamento bidirecional do \gls{bert} é alcançado por meio de duas técnicas, sendo elas \gls{mlm} e \gls{nsp}. \gls{mlm} mascara parte das palavras em uma frase e então o modelo é treinado para prever essas palavras com base no contexto ao redor, tanto antes quanto depois da posição mascarada. \gls{nsp} treina o modelo para determinar se uma frase B segue imediatamente uma frase A no texto, melhorando a compreensão em relação às relações entre sentenças. \cite{Devlin2018}.

As principais caracteristicas do \gls{bert} são sua bidirecionalidade e escalabilidade, sendo uma de suas versões, o BERT-Large, composto por 24 camadas, 1024 dimensões e 340 milhões de parâmetros. A partir de seu lançamento, o BERT tornou-se base para diversos avanços e está presente em aplicações práticas, como por exemplo o sistemas de busca do Google. \cite{Devlin2018}.

\subsection{BERTimbau}\label{subsec:bertimbau}

% texto

\subsection{Azure Analytics}\label{subsec:azure}

% https://learn.microsoft.com/en-us/azure/ai-services/language-service/?source=recommendations

\subsection{Amazon Comprenhend}\label{subsec:comprenhend}

% texto

\subsection{YAKE}\label{subsec:yake}

% texto

\section{Métodos}\label{sec:metodo}

\subsection{Marcação de Parte de Fala}\label{subsec:pos_tagging}

% https://web.stanford.edu/~jurafsky/slp3/ed3bookaug20_2024.pdf

\subsection{Extração de Palavras-Chave}\label{subsec:keyword_extraction}

% texto

\subsection{Similaridade do Cosseno}\label{subsec:cossine_similarity}

% texto