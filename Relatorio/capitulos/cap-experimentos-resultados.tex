% ------------------------------------------------------------------------------------------- %
%                                    Resultados e Discussão                                   %
% ------------------------------------------------------------------------------------------- %
\chapter{Resultados}\label{cap:resultados}

Os resultados decorrentes do desenvolvimento deste projeto foram:

\begin{itemize}
	\item Dados resultantes da comparação entre cinco diferentes inteligências e seu desepenho na extração de contextos relevantes presentes em dados de laboratórios e demandas coletados pelo \gls{direc} ao longo dos anos.
	\item A criação de uma aplicação capaz de integrar diferentes modelos de inteligência artificial para a análise linguistica de textos e extração de palavras-chave, bem como a análise de similiaridade entre contextos.
	\item A criação de uma aplicação capaz de mediar a utilização de uma aplicativo mobile com uma aplicação de análise de dados.
	\item Um \gls{mvp} mobile como prova de conceito na utilização das tecnologias empregadas, aceitando a entrada de dados de usuários e oferencendo resultados compreensíveis.
\end{itemize}

% ------------------------------------ Escopo do Sistema ------------------------------------ %
\section{Escopo do sistema}\label{sec:escopoSistema}

O projeto Contrate um Cientista possui dois objetivos principais, sendo eles, a realização de uma análise comparativa entre diferentes modelos de linguagem na terefa de associação de demandas com laboratórios, e a facilitação da formação de parcerias entre o setor produtivo e a expertise científica presente nas universidades através de uma aplicação que, de forma inteligente e automatizada, norteie a tomada de decisões na formação de parcerias. Com estes objetvios postos, é necessaário realizar algumas considerações sobre o escopo do projeto.

O sistema desenvolvido não visa eliminar a necessidade de avaliação das sugestões realizadas pelos profissíonais responsáveis, mas sim fornecer uma ferramenta que auxilie na tomada de decisão, pois nenhum sistema é a prova de falhas. O aplicativo desensolvido é uma prova de conceito e não deve ser visto como uma solução pronta para ambiente de produção, pois mesmo que implemente um mecânismo básico de segurança, não foi desenvolvido com foco na segurança dos dados. Para um cenário de utilização real, a base de dados deve estar presente é um ambiente seguro com acesso controlado, e o armazenamento de dados de usuários deve seguir as regras impostas pela \gls{lgpd}.  

As inteligências \gls{bert}, \gls{gpt}, Azure AI Language e Amazon Comprehend foram escolhidas por se tratar de modelos de linguagem desensolvidos pelas quatro gigantes na área de tecnologia e \gls{ia}, Google, OpenAI, Microsoft e Amazon, respectivamente. A inteligência \gls{yake}, mesmo sendo um modelo análitico, foi escolhido para efeitos de comparação entre \gls{llm} e modelos de análise de texto mais tradicionais. Outro fator que contribuiu para a escolha destas tecnologias foi a disponibilidade de uma \gls{api} de integração e facilidade de implementação.

O sistema abrange dois tipos de usuários, organizações e laboratórios. Ambos os tipos de usuário utilizaram a mesma tela de login para acesso à aplicação, sendo que o conteúdo apresentado após autenticação no sistema difere a depender do tipo de usuário. Organizações são responsáveis pela criação das demandas, cujo conteúdo é analisado contra as informações de todos os laboratórios cadastrados, gerando uma pontuação indicativa de similaridade para cada um. Organizações tambem seram capazes de favoritar os laboratórios de sua preferência por demanda. Laboratórios são responsáveis pelo cadastro de seus recursos relavantes, como descrição do trabalho que efetuam, equipamentos, softwares e certificados, que serão utilizados para a análise e podem ser visualizados pelas organizações.

Para realizar a comparação entre as inteligências integradas foi construído um projeto a parte de teste e benchmarking. Este projeto simula o consumo dos pontos de extremidade expostos pela \gls{api} de aplicação da mesma forma que o aplicativo mobile, porém obtendo informações relevantes para à análise de desempenho, como por exemplo, o tempo de análise de cada inteligência, armazenando as informações obtidas em um banco de dados. Estas informações são utilizadas na conclusão e ajudam a enteder os pontos positivos e possíveis melhorias que podem basear trabalhos futuros.

% ----------------------------------- Modelagem do Sistema ---------------------------------- %
\section{Modelagem do sistema}\label{sec:modelagemSistema}

A maior dificuldade atualmente, se concentra nos responsáveis do departamento que faz esse match entre uma demanda e um cientista nas universidades.
Pensando nisso, o sistema deveria ser de fácil acesso à esses responsáveis, por isso, o App foi uma solução que melhor se enquadraria. 
Finalmente, foi realizado um diagrama para que seja facilmente entendido quais seriam organizados os projetos necessários, juntamente com suas tecnologias, conforme \autoref{fig:fluxograma-comunicacao}.

\begin{figure}[htb]
    \captionsetup{width=0.43\textwidth}
    \caption{Diagrama da plataforma.}
    \label{fig:fluxograma-comunicacao}
    \includegraphics[scale=0.8]{fluxograma-comunicacao}
    \fonte{}
\end{figure}

Para representar os tipos de usuários e suas possíveis funções, temos os diagramas de casos de uso das \autoref{fig:caso-uso-org} e \autoref{fig:caso-uso-lab}. 

A \autoref{fig:caso-uso-org} mostra as ações que as organizações podem realizar, que são: se cadastrar na plataforma, visualizar e editar informações do perfil, criar demandas com suas especifidades, visualizar todas as demandas previamente criadas, visualizar todos os matches para cada demanda, assim como as informações dos respectivos laboratórios e favoritar esses laboratórios.

\begin{figure}[htb]
    \captionsetup{width=0.43\textwidth}
    \caption{Diagrama de caso de uso da organização.}
    \label{fig:caso-uso-org}
    \includegraphics[scale=0.8]{caso-uso-org}
    \fonte{}
\end{figure}

O diagrama de caso de uso da \autoref{fig:caso-uso-lab}, mostra as açoes que os labartoratórios podem realizar: se cadastrar, entrar na plataforma, visualizar e editar informações do perfil e visualizar as demandas que a organização proprietária o favoritou.

\begin{figure}[htb]
    \captionsetup{width=0.43\textwidth}
    \caption{Diagrama de caso de uso da organização.}
    \label{fig:caso-uso-lab}
    \includegraphics[scale=0.8]{caso-uso-lab}
    \fonte{}
\end{figure}

O banco de dados foi planejado para comportar os dois tipo de usuários diferentes (laboratório e organização), e para armazenar todos os dados relevantes para análise das demandas, de acordo com o diagrama da \autoref{fig:ERD}.

\begin{figure}[h!]
    \captionsetup{width=0.43\textwidth}
    \caption{Diagrama de Relação de Entidades do Bando de Dados.}
    \label{fig:ERD}
    \includegraphics[scale=0.5]{ERD}
    \fonte{}
\end{figure}

Como a função principal, é a criação de demandas, é possível entender melhor os passos através do diagrama \autoref{fig:diagrama-sequencia}, que exibe os passos desde a criação de demanda, até a criação da relação entre a demanda e os laboratórios.

A organização, através do aplicativo, pode criar a demanda, e após preencher todos os dados da demanda, e salvar, esses dados serão enviados para a \gls{api} de aplicação. A \gls{api} de aplicação irá validar os dados, salvar no banco de dados e então fazer uma requisição para a \gls{api} de linguagem, para que ela salve as palavras chaves adicionais da demanda em questão, e faça o match com todos os laboratórios ja cadastrados. Após encontrados os laboratórios que deram match com a demanda, ele irá relacionar aquela demanda com o laboratório no banco de dados.

\begin{figure}[htb]
    \captionsetup{width=0.43\textwidth}
    \caption{Diagrama de sequência de criação de demanda.}
    \label{fig:diagrama-sequencia}
    \includegraphics[scale=0.8]{diagrama-sequencia}
    \fonte{}
\end{figure}

\renewcommand{\labelenumii}{\arabic{enumi}.\arabic{enumii}}
\renewcommand{\labelenumiii}{\arabic{enumi}.\arabic{enumii}.\arabic{enumiii}}
\renewcommand{\labelenumiv}{\arabic{enumi}.\arabic{enumii}.\arabic{enumiii}.\arabic{enumiv}}

\subsection{Requisitos Funcionais}\label{subsec:rf}
\begin{enumerate}
    \item O sistema deve disponibilizar um aplicativo
    \item O aplicativo deve ter uma página para o cadastro das organizações
	\begin{enumerate}
		\item Organização deve ter um email para login
		\item Organização deve ter uma senha
		\item Organização deve ter um nome
		\item Organização deve ter um cnpj
		\item Organização deve ter um email para contato
		\item Organização deve permitir uma descrição
	\end{enumerate}
    \item O aplicativo deve ter uma página para o cadastro de laboratórios
	\begin{enumerate}
		\item Laboratório deve ter um email para login
		\item Laboratório deve ter um código
		\item Laboratório deve ter uma data de fundação
		\item Laboratório deve ter palavras-chave
		\item Laboratório deve permitir uma descrição
		\item Laboratório deve permitir o cadastro dos seus certificados
		\item Laboratório deve ter um responsável
		\begin{enumerate}
			\item Responsável deve ter um nome
			\item Responsável deve permitir um departamento
			\item Responsável deve permitir um email
			\item Responsável deve permitir um telefone
		\end{enumerate}
		\item Laboratório deve ter um endereço
		\begin{enumerate}
			\item Endereço deve ter uma rua
			\item Endereço deve ter um número
			\item Endereço deve ter um bairro
			\item Endereço deve ter uma cidade
			\item Endereço deve ter um estado
			\item Endereço deve ter um país
			\item Responsável deve permitir um complemento
		\end{enumerate}
		\item Laboratório deve permitir uma descrição
		\item Laboratório deve permitir uma ou mais redes sociais
		\begin{enumerate}
			\item Rede social deve ter um tipo
			\item Rede social deve ter um link
		\end{enumerate}
		\item Laboratório deve permitir um ou mais equipamentos
		\begin{enumerate}
			\item Equipamento deve ter um nome
			\item Equipamento deve permitir uma descrição
			\item Equipamento deve permitir uma área
		\end{enumerate}
		\item Laboratório deve permitir um ou mais softwares
		\begin{enumerate}
			\item Software deve ter um nome
			\item Software deve permitir uma descrição
			\item Software deve permitir uma área
		\end{enumerate}
	\end{enumerate}
    \item O aplicativo deve ter uma página de login para os usuários
    \item O aplicativo deve permitir que o usuário faça o logout da sua conta
    \item O sistema deve permitir que um usuário visualize e edite as informações do seu perfil no sistema
    \item O aplicativo deve permitir que uma organização crie uma demanda
	\begin{enumerate}
		\item Demanda deve ter um título
		\item Demanda deve ter um departamento
		\item Demanda deve ter benefícios
		\item Demanda deve ter detalhes
		\item Demanda deve ter palavras-chave
		\item Demanda deve permitir uma descrição
		\item Demanda deve permitir restrições
	\end{enumerate}
	\item O sistema deve permitir que uma organização visualize todas suas demandas no sistema
	\item O sistema deve permitir que uma organização favorite laboratórios aptos para uma demanda específica
	\item O sistema deve permitir que uma organização visualize os laboratórios favoritados para cada demanda
	\item O sistema deve permitir que um laboratório visualize as informações da demanda após aquela organização o marcar como favorito
\end{enumerate}

\subsection{Requisitos Não-Funcionais}\label{subsec:rnf}
\begin{enumerate}
\item O sistema deve permitir ser instalado em sistema operacionais Android e iOs
\item O sistema deve ser usado a framework Dart
\item O sistema deve ser servido em C\#
\item O site deverá armazenar dados persistentes com PostgreSQL.
\item O sistema deve armazenar a senha criptografada
\item O sistema deve manter todas as informações protegidas de acordo com permissões do usuário logado
\item O sistema deve mostrar a organização os melhores candidatos para trabalhar naquela demanda
\item O sistema não deve travar até a análise de todos os laboratórios para aquela nova demanda
\end{enumerate}


% --------------------------------- Apresentação do Sistema --------------------------------- %
\section{Apresentação do sistema}\label{sec:apresentacaoSistema}

Para uma representação visual do fluxo e da estrutura de interação entre todas as páginas do sistema, é utilizado o diagrama de navegação na \autoref{fig:diagrama-navegacao}, onde mostra como os usuários navegarão de uma tela para outra.

\begin{figure}[h!]
    \captionsetup{width=0.43\textwidth}
    \caption{Diagrama de navegação de telas do aplicativo.}
    \label{fig:diagrama-navegacao}
    \includegraphics[scale=0.8]{diagrama-navegacao}
    \fonte{}
\end{figure}

O sistema possui uma tela para entrar inicialmente, onde o usuário pode preencher seu email e senha, ou fazer seu registro. Antes de realizar o signin, o usuário precisa escolher se ele é um laboratório ou organização a se registrar, e então ele é direcionado para seu respectivo formulário para preencher com suas informações. As telas para entrar e se cadastrar nos sistemas estão representadas na \autoref{fig:pagina-login-signin}.

\begin{figure}[h!]
    \captionsetup{width=0.43\textwidth}
    \caption{Tela para entrar e se registrar do aplicativo.}
    \label{fig:pagina-login-signin}
    \includegraphics[scale=0.6]{pagina-login-signin}
    \fonte{}
\end{figure}

Após realizar o login e senha, o usuário será redirecionado para sua tela principal, á esquerda da \autoref{fig:pagina-home} está a tela para um usuário do tipo organização, e á direita, para um usuário do tipo laboratório.

\begin{figure}[h!]
    \captionsetup{width=0.43\textwidth}
    \caption{Telas iniciais de um usuário do tipo organização e laboratório, respectivamente, do aplicativo.}
    \label{fig:pagina-home}
    \includegraphics[scale=0.6]{pagina-home}
    \fonte{}
\end{figure}

Ou seja, uma organização poderá ver seu perfil, as demandas e os matches de todas as demandas, além dos seus detalhes, demonstrado pela, enquanto o laboratório poderá ver seu perfil e os matches com os detalhes da demanda do organizador que o marcou como favorito.

A tela de perfil, é onde o usuário poderá visualizar e editar todas as suas informações de perfil, na \autoref{fig:pagina-form-perfil} temos essas telas de organização e laboratório, respectivamente. Esse mesmo formulário, é visível ao se registrar no aplicativo.

\begin{figure}[h!]
    \captionsetup{width=0.43\textwidth}
    \caption{Tela de detalhes de perfil da organização e laboratório, respectivamente, do aplicativo.}
    \label{fig:pagina-form-perfil}
    \includegraphics[scale=0.4]{pagina-form-perfil}
    \fonte{}
\end{figure}

As páginas das demandas na \autoref{fig:paginas-demanda}, podem ser gerenciadas apenas para usuários do tipo organização. E é possível visualizar uma lista com todas as demandas já cadastradas, editar a demanda apenas se nenhum laboratório estiver associada a ela, visualizar os detalhes da demanda, e criar uma demanda nova.

\begin{figure}[h!]
    \captionsetup{width=0.43\textwidth}
    \caption{Telas de lista de demandas, detalhe de demanda e criação de demanda, respectivamente, do aplicativo.}
    \label{fig:paginas-demanda}
    \includegraphics[scale=0.4]{paginas-demanda}
    \fonte{}
\end{figure}

Uma lista comum para ambos os usuários, é a tela de lista dos matches na \autoref{fig:pagina-matches}, onde é pode-se ver os detalhes daquela match.

\begin{figure}[h!]
    \captionsetup{width=0.43\textwidth}
    \caption{Tela da lista de matches do aplicativo.}
    \label{fig:pagina-matches}
    \includegraphics[scale=0.6]{pagina-matches}
    \fonte{}
\end{figure}

Para o usuário organização, ao clicar para visualizar os detalhes do match, ele tem acesso a todas as informações do laboratório que deu match com aquela demanda, e as descrições de equipamentos e softwares são visiveis através de um modal. Além dos detalhes, a organização pode marcar aquele laboratório como apto para aquela demanda, o que faz com que o no aplicativo do laboratóio, ele possa ver os detalhes dessa demanda na respectiva tela. Já para o usuário laboratório, o botão de detalhes o leva para visualizar as informações completas da demanda. Ambas as página são mostradas na \autoref{fig:pagina-match}, e o título das duas páginas de detalhes, é o título da demanda em questão.

\begin{figure}[h!]
    \captionsetup{width=0.43\textwidth}
    \caption{Telas de detalhes de um match, com os detalhes do laboratório, assim como a descrição do equipmanto, de um usuário do tipo organização do aplicativo. E a tela de detalhe da demanda visualizado no aplicativo de laboratórios, com detalhes da demanda.}
    \label{fig:pagina-match}
    \includegraphics[scale=0.4]{pagina-match}
    \fonte{}
\end{figure}

% --------------------------------- Implementação do Sistema -------------------------------- %
\section{Implementação do sistema}\label{sec:implementacaoSistema}

% Nesta seção é documentada a implementação do sistema com partes relevantes ou exemplos de código, rotinas, funções. Inclui, ainda, a descrição técnica do uso de recursos (componentes, bibliotecas, etc.) da linguagem. Ressalta-se que cada orientador avaliará juntamente com seu orientado o que poderá ser descrito nesta seção. Isso sem que sejam revelados detalhes do sistema que possam comprometer seu uso comercial ou científico ou que a descrição fique muito sucinta ou superficial.

\subsection{Aplicação}\label{subsec:aplicacao}
A aplicação foi desenvolvida em Flutter, um framework, que utiliza o Dart como linguagem de programação. A interface é construída com componentes de interface para compor as páginas. Os componentes foram criados e organizados de forma independente, visando facilitar futuras implementações e alterações. Como por exemplo na \autoref{codigo:equip-form}, o componente de formulário do equipamento, que pode ser utilizado em diferentes páginas. O formulário é utilizado tanto para edição, quando para criação de equipamentos, e possui o campo "Nome" obrigatório, enquanto os campos "Descrição" e "Área" são opcionais. Quando pressionado o botão para salvar, o modal é fechado e os controllers são retornados populados para a página de lista de equipamentos, que salva os dados no banco de dados.

\begin{sourcecode}[htb]
    \caption{\label{codigo:equip-form}Componente do formulário de equipamentos}
    \begin{lstlisting}[frame=single, language=Java]
class EquipmentFormPage extends StatelessWidget {
  EquipmentFormPage(
      {Key? key,
      required this.nameController,
      required this.descriptionController,
      required this.areaController})
      : super(key: key);

  ...

  @override
  Widget build(BuildContext context) {
    return Scaffold(
      body: Form(
        key: _formKey,
        child: Column(mainAxisAlignment: MainAxisAlignment.center, children: [
          const Text('Nome *'),
          Padding(
            padding: const EdgeInsets.symmetric(horizontal: 16, vertical: 16),
            child: TextFormField(
              decoration: const InputDecoration(border: OutlineInputBorder()),
              controller: nameController,
              validator: (value) {
                if (value == null || value.isEmpty) {
                  return 'Por favor, insira um nome';
                }
                return null;
              },
            ),
          ),
		  
          ...
		  
          Padding(
            padding: const EdgeInsets.symmetric(horizontal: 16, vertical: 16),
            child: ElevatedButton(
              child: const Text('Salvar'),
              onPressed: () {
                Navigator.pop(context, true);
              },
            ),
          ),

\end{lstlisting}
    \fonte{}
\end{sourcecode}

Para a comunicação com a API de Aplicação, foi utilizada o pacote nativo http, onde é possível fazer todas as requições, com todos os parâmetros necessários. Entre esses parâmetros, está o token utilizado para a autenticação da rota. O token é registrado ao realizar o login no armazenamento interno, utilizando o pacote Shared Preferences, e sendo possível buscá-lo em todas as requisições. Abaixo na \autoref{codigo:list-demands}, é possível visualizar um trecho do código utilizada para buscar as desmandas da organização logada.

\begin{sourcecode}[htb][h!]
    \caption{\label{codigo:list-demands}Buscar lista de demandas}
    \begin{lstlisting}[frame=single, language=Java]
static Future<List<Demand>?> getDemand() async {
    try {
      var url = Uri.parse(
          '${ApiConstants.baseUrl}${ApiConstants.demandEndpoint}/list');
		  
	  // buscar token
      final SharedPreferences prefs = await SharedPreferences.getInstance();
      final String? token = prefs.getString('token');
	  
	  // faz requisição das demandas daquela organização
      var response =
          await http.get(url, headers: {'Authorization': 'Bearer $token'});
      if (response.statusCode == 200) {
        var body = json.decode(response.body);
		// retorna a lista de demandas
        return List<Demand>.from(body.map((item) => Demand.fromMap(item)));
      }
    } catch (e) {
	  throw Exception(e.toString());
    }
}
\end{lstlisting}
    \fonte{}
\end{sourcecode}

% Em materiais e método estão quais os recursos utilizados, neste capítulo é reportado como esses recursos foram utilizados para resolver o problema.
\subsection{API de Aplicação}\label{subsec:api-aplicacao}

\subsection{API de linguagem}\label{subsec:api-linguagens}


% Enfatizar os diferenciais do sistema: procedimentos armazenados, consultas SQL, uso de componentes, uso de padrões de projeto, a forma de uso dos recursos da linguagem. Esses diferenciais são no sentido de explicitar as vantagens, desvantagens, dificuldades e facilidades que esses recursos impetraram no desenvolvimento do sistema em termos técnicos. Esses diferenciais servirão para avaliar pela utilização ou não desses recursos, pelo menos para sistemas iguais ou semelhantes ao reportado no trabalho.

% Reportar a forma como o sistema foi verificado e validado. No sentido de verificar se os requisitos definidos para o mesmo foram atendidos. Os testes podem ser realizados pelo professor orientador, pelos professores que compõem a banca, por pessoas que serviram de base para as informações para o sistema e etc. Os testes podem ser realizados com base em um plano de testes elaborado juntamente com a análise e projeto do sistema. Para validar a implementação podem ser desenvolvidas rotinas de teste unitário.

Por falta de disponibilidade de um equipamento com sistema operacional macOS, o aplicativo foi desenvolvido apenas para a plataforma Android. Atualmente, ele conta com um arquivo instalador que precisa ser transferido para o celular para permitir a instalação direta no dispositivo. No futuro, se faz necessário disponibilizá-lo na Google Play Store, a fim de simplificar o processo de instalação e oferecer maior praticidade aos usuários

