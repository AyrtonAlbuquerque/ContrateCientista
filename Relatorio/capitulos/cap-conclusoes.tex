% ------------------------------------------------------------------------------------------- %
%                                          Conclusões                                         %
% ------------------------------------------------------------------------------------------- %
\chapter{Conclusão}\label{cap:conclusoeseperspectivas}

Inicia com um resumo do trabalho, retomando o(s) objetivo(s), o referencial teórico e o uso das ferramentas e das tecnologias utilizadas no trabalho.

A conclusão contém a opinião do autor em relação às vantagens, desvantagens, facilidades e limitações das tecnologias e/ou do método utilizados, as dificuldades encontradas e como foram superadas.

Também devem ser apresentadas as vantagens, desvantagens e limitações do trabalho desenvolvido, sempre tendo em vista a sua contribuição para a comunidade acadêmica e profissional e para a sociedade como um todo.

É a opinião técnica do autor do trabalho em relação ao assunto sob a forma de uma espécie de avaliação em relação ao trabalho desenvolvido e as tecnologias utilizadas.

Finaliza verificando se o objetivo foi alcançado e com a opinião do autor sobre o assunto, de acordo com o referencial teórico e com os resultados obtidos.

As perspectivas futuras são opcionais, devem ser apresentadas somente caso o acadêmico pretenda dar continuidade ao trabalho, ou mesmo se ele julgar relevante que outras pessoas dêem continuidade ao seu trabalho.
