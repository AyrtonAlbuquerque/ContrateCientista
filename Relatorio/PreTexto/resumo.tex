% ------------------------------------------------------------------------------------------- %
%                                            Resumo                                           %
% ------------------------------------------------------------------------------------------- %
\begin{resumoutfpr}
    O processo de busca por inovações tecnológicas e científicas geralmente passa pelas universidades, onde boa parte das pesquisas de ponta é feita. Apesar da excelência na criação de novas soluções, o processo de vínculo entre uma demanda tecnológica advinda de organizações públicas ou privadas e as universidades ou ICTs para se chegar até um cientista capaz de atender tal demanda, é por muitas vezes lento, ineficiente e burocrático, o que acaba desencorajando a procura das ICTs para formar parcerias. Para atender esse problema, o projeto visa buscar um matching entre uma demanda específica do setor produtivo e comercial e a expertise e conhecimento disponível no setor acadêmico. Este matching requer informações relevantes sobre ambos, expertise científica do setor acadêmico, bem como o entendimento do problema e dos recursos que poderão estar disponíveis para encontrar a solução demandada. Aqui entende-se por matching, a apresentação de possíveis melhores candidatos (cientistas) aptos a auxiliar no entendimento e/ou criação de soluções específicas para a demanda proposta pelo setor produtivo. Uma segunda etapa de matching seria a de levantamento de recursos iniciais necessários para o estudo e desenvolvimento das soluções e/ou geração de conhecimento. Uma fonte importante sobre a expertise científica brasileira é a plataforma \emph{CV Lattes}, plataforma governamental que armazena os dados de pesquisadores e profissionais brasileiros. No entanto, as informações sobre o problema do setor produtivo nem sempre são apresentados de maneira a proporcionar uma rápida associação com as informações cadastradas no \emph{CV Lattes}.
    Um conceito interessante da computação é o de dados conectados, que são dados estruturados e interligados a outros dados que podem ser relacionados e interpretados por máquinas e que tornam a consulta semântica mais útil e simples. Este conceito pode ser interessante para relacionar termos que se associem à demanda, sugerir novos termos, estabelecer modelos de filtragem baseados em interesses e inovar procurando entender novas relações que possam futuramente gerar novas conexões de dados. A contribuição deste TCC está em aplicar conceitos de dados conectados para uma aplicação de matching sugerida pela \emph{Diretoria de Relações Empresariais e Comunitárias} (DIREC) no projeto \emph{Contrate um Cientista}. A validação do projeto será feita através do DIREC em conjunto com empresas já parceiras, e membros do corpo acadêmico da \emph{Universidade Tecnológica Federal do Paraná} (UTFPR), comparando resultados com ferramentas usualmente utilizadas para atingir o mesmo resultado.
\end{resumoutfpr}