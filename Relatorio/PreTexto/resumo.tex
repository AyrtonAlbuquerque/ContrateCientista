% ------------------------------------------------------------------------------------------- %
%                                            Resumo                                           %
% ------------------------------------------------------------------------------------------- %
\begin{resumoutfpr}
    O processo de busca por inovações tecnológicas e cientificas geralmente passa pelas universidades, onde muitas das mentes mais brilhantes de nosso tempo se encontram e onde boa parte das pesquisas de ponta é feita. Apesar da excelência na criação de novas soluções, o processo de vínculo entre uma demanda tecnológica advinda de organizações públicas ou privadas e as universidades ou ICTs para se chegar até um cientista capaz de atender tal demanda, é por muitas vezes lento, ineficiente e burocrático, o que acaba por desencorajar o mundo produtivo em procurar as ICTs para formar parcerias. Para atender esse problema, o projeto \emph{Contrate um Cientista} tem como objetivo criar uma plataforma onde o setor produtivo, comercial e acadêmico, por meio de uma aplicação, poderá encontrar um cientista capaz de atender suas demandas de forma mais simples e direta, sem a necessidade de terceiros.
    A aplicação utilizará a base de dados do \emph{CV Lattes}, plataforma governamental que armazena os dados de pesquisadores e profissionais brasileiros. Outra estratégia é possibilitar que os pesquisadores possam criar seu prórpio cadastro dentro da plataforma. O objetivo principal deste projeto é integrar esta ferramenta a uma aplicação, capaz de proporcionar a um usuário a possibilidade de encontrar um cientista em uma área especifica de conhecimento, que seja capaz de atender suas demandas tecnológicas e conectar o usuário diretamente ao pesquisador, facilitando assim, a formação de parcerias e contratos que beneficiem tanto o usuário, quanto a instituição a qual o cientista serve.
\end{resumoutfpr}
