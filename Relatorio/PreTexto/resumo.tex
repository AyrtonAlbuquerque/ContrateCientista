% ------------------------------------------------------------------------------------------- %
%                                            Resumo                                           %
% ------------------------------------------------------------------------------------------- %
\begin{resumoutfpr}
    O processo de busca por inovações tecnológicas e científicas geralmente passa pelas universidades, onde boa parte das pesquisas de ponta é feita. Apesar da excelência na criação de novas soluções, o processo de vínculo entre uma demanda tecnológica advinda de organizações públicas ou privadas e as universidades ou ICTs para se chegar até um cientista capaz de atendê-la, é por muitas vezes lento, ineficiente e burocrático, o que acaba desencorajando a procura das ICTs para formar parcerias. Para atender esse problema, o projeto visa buscar um matching entre uma demanda específica do setor produtivo e comercial, e a expertise e conhecimento disponíveis no setor acadêmico. Este matching requer informações relevantes sobre ambos, expertise científica do setor acadêmico, bem como o entendimento do problema e dos recursos que poderão estar disponíveis para encontrar a solução demandada. Aqui entende-se por matching, a apresentação de possíveis melhores candidatos (laboratórios) aptos a auxiliar no entendimento e/ou criação de soluções específicas para a demanda proposta pelo setor produtivo. A principal fonte de expertise científica utilizada neste projeto são dados de laboratórios de pesquisa da \emph{Universidade Tecnológica Federal do Paraná} (UTFPR) mantidos pelo \emph{Diretoria de Relações Empresariais e Comunitárias} (DIREC). No entanto, as informações sobre o problema do setor produtivo nem sempre são apresentados de maneira a proporcionar uma rápida associação com as informações cadastradas no \emph{DIREC}.
    Um conceito interessante da computação é o de \emph{Grande Modelos de Linguages} (LLM), sendo modelos de \emph{Inteligência Artificial} (AI) construídos a partir de técnicas de \emph{Aprendizado de Máquina} (ML) e \emph{Processamento de Linguagem Natural} (NLP) para interpretar e trabalhar com um variado conjunto de dados. Este conceito pode ser interessante para relacionar termos que se associem à demanda, sugerir novos termos, estabelecer modelos de filtragem baseados em interesses e inovar procurando entender novas relações que possam futuramente gerar novas conexões de dados. A contribuição deste TCC está em realizar uma análise comparativa entre alguns dos grandes \emph{LLMs} da indústria e em como cada um se sai ao analisar dados de laboratórios e demandas para extrair dados relevantes e criar associações entre eles.
\end{resumoutfpr}