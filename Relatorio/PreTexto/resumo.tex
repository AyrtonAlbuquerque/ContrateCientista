% ------------------------------------------------------------------------------------------------ %
%                                              Resumo                                              %
% ------------------------------------------------------------------------------------------------ %
\begin{resumoutfpr}
    O processo de busca por inovações tecnológicas e ciêntificas sempre passa pelas universidades, onde muitas das mentes mais brilhantes de nosso tempo se encontram e onde boa parte das pesquisas de ponta é feita. Apesar da excelência na criação de novas soluções, o processo de vínculo entre uma demanda tecnológica e um pesquisador capaz de atender tal demanda, é por muitas vezes lento, ineficiente e burocrático, o que acaba por desencorajar o mundo produtivo em procurar as instituições para formar parcerias. Para vir em socorro desse problema, o projeto "Contrate um Ciêntista" foi idealizado, uma plataforma onde o setor produtivo, por meio de uma aplicação, poderá encontrar um pesquisador capaz de atender suas demandas de forma mais simples e direta, sem a necessidade de terceiros.

    Para alcançar este objetivo, a aplicação utilizará a base de dados do "CV Lattes", plataforma governamental que armazena os dados de pesquisadores e profissionais brasileiros. O objetivo principal deste projeto, é integrar esta ferramenta a uma aplicação, capaz de proporcionar um usuário a possibilidade de encontrar um pesquisador em uma área especifica de conhecimento, que seja capaz de atender suas demandas tecnológicas e conectar o usuário diretamente ao pesquisador, facilitando assim, a formação de parcerias e contratos que beneficiem tanto o usuário, quanto a instituição a qual o pesquisador serve.
\end{resumoutfpr}
