% ------------------------------------------------------------------------------------------- %
%                                           Abstract                                          %
% ------------------------------------------------------------------------------------------- %
\begin{abstractutfpr}
    The process of searching for cientific and technological innovation usually goes through universities, where most of the brilliant minds of our time are and where a great portion of the cutting edge research is done. Despite the excelence in the creation of new solutions, the process of linking a technological demand from public or private organizations and universities or ICTs to find a cientist able to meet such demand is often slow, inefficient and bureaucratic, which ends up discouraging the productive world from looking for ICTs to form partnerships. For this problem the project aims to seek a match between a specific demand of the productive and commercial sector and the expertise and knowledge available in the academic sector. This one correspondence requires relevant information about both academic sector scientific expertise as well as an understanding of the problem and the resources that may be available to find a required solution. Here it is understood by matching, a presentation of possible best candidates (laboratories) able to assist in understanding and/or creating solutions specific to the demand proposed by the productive sector. The main source of scientific expertise used in the project is data from reseasrch laboratories of \emph{Universidade Tecnológica Federal do Paraná} (UTFPR) maintained by \emph{Board of Business and Community Relations} (DIREC). However, information about the industry problem produced are not always presented in such a way as to provide a quick association with the information registered in \emph{DIREC}.
    An interesting concept in computing is that of \emph{Large Language Models} (LLM), which are \emph{Artificial Intelligence} (AI) models built using \emph{Machine Learning} (ML) and \emph{Natural Language Processing} (NLP) techniques to interpret and work with a diverse set of data. This concept can be interesting to relate terms that are associated with demand, suggesting new terms, establishing filtering models based on interests and innovate by seeking to understand new relationships that may in the future generate new connections of data. The contribution of this thesis is to perform a comparative analysis of some of the major \emph{LLMs} in the industry and how each one performs in analyzing laboratory data and demands to extract relevant information and create associations between them.
\end{abstractutfpr}
