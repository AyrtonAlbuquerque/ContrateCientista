% ------------------------------------------------------------------------------------------- %
%                                           Abstract                                          %
% ------------------------------------------------------------------------------------------- %
\begin{abstractutfpr}
    The process of searching for cientific and technological innovation usually goes through universities, where most of the brilliant minds of our time are and where a great portion of the cutting edge research is done. Despite the excelence in the creation of new solutions, the process of linking a technological demand from public or private organizations and universities or ICTs to find a cientist able to meet such demand is often slow, inefficient and bureaucratic, which ends up discouraging the productive world from looking for ICTs to form partnerships. To address this problem the \emph{Hire a Cientist} project aims to create a platform where the productive, commercial and academic sectors can, through an application, search for a cientist able the meet its demands in a simple and straight foward way, without the need for third parties.
    The application utilizes the informations registered by the schientist, along with the \emph{CV Lattes} database, a goverment platform the stores the information about brazilian professionals and researchers. The expected results of this project is to integrate this tool with an application capable os providing to a user the possibility of finding a cientist in a especific area of knowledge that is able to meet a technological demand and connect the user directly to this cientist, facilitatig the creation of partnerships and contracts that benefits the user as well as the institution at which the cientist works on. The validation of the project will be done through \emph{DIREC} together with companies that are already partners, and members of the academic body of \emph{UTFPR}, and also comparing results with \emph{LinkedIn}, \emph{Escavador} and \emph{CV Lattes} platforms
\end{abstractutfpr}
