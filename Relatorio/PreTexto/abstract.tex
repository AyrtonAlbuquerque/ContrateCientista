% ------------------------------------------------------------------------------------------- %
%                                           Abstract                                          %
% ------------------------------------------------------------------------------------------- %
\begin{abstractutfpr}
    The process of searching for cientific and technological innovation usually goes through universities, where most of the brilliant minds of our time are and where a great portion of the cutting edge research is done. Despite the excelence in the creation of new solutions, the process of linking a technological demand from public or private organizations and universities or ICTs to find a cientist able to meet such demand is often slow, inefficient and bureaucratic, which ends up discouraging the productive world from looking for ICTs to form partnerships. For this problem the project aims to seek a match between a specific demand of the productive and commercial sector and the expertise and knowledge available in the academic sector. This one correspondence requires relevant information about both academic sector scientific expertise as well as an understanding of the problem and the resources that may be available to find a required solution. Here it is understood by matching, a presentation of possible best candidates (scientists) able to assist in understanding and/or creating solutions specific to the demand proposed by the productive sector. A second matching step would be raising initial funds for the study and development of solutions and/or generation of knowledge. An important source on scientific expertise Brazilian is the \emph{CV Lattes} platform, a government platform that stores the data of Brazilian researchers and professionals. However, information about the industry problem produced are not always presented in such a way as to provide a quick association with the information registered in the \emph{CV Lattes}.
    An interesting concept in computing is the linked data, which are structured data interlinked with other data that can be related and interpreted by machines. This concept can be interesting to relate terms that are associated with demand, suggesting new terms, establishing filtering models based on interests and innovate by seeking to understand new relationships that may in the future generate new connections of data. The contribution of this TCC is to apply linked data concepts to an application in correspondence suggested by the \emph{Board of Business and Community Relations} (DIREC) in the project \emph{Hire a Scientist}. The validation of the project will be done through \emph{DIREC} together with companies that are already partners, and members of the academic body of \emph{UTFPR}, comparing results with the tools that are usually used to achieve the same results.
\end{abstractutfpr}
